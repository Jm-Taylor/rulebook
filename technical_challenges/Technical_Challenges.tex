% !TEX root = ../Rulebook.tex

In the medium term, the \RCAW aims to transfer specific aspects of industrial scenarios in the tests and to demonstrate the practical applicability of the solutions. The challenges, which are adapted or redefined annually, serve as a test platform for the further development of the competitions. Each technical challenge is separately awarded. That means, teams can participate in any number of them. Hoewever any challenge will only be awarded if at least two teams competed unless the only competing team provided an outstanding performance.

A challenge increases the level capabilities of a robot in \RCAW related to:

\begin{itemize}
  \item \textbf{Variability of the environmental conditions} ... The setup conditions of a run are designed variably including disturbances. The lighting situations in the arena are changed dynamically, the configuration of the tables (height, format) is adapted or manipulation objects are mixed with unknown decoy objects.
  \item \textbf{Complexity of the scenarios} ... New arena elements are involved in a scenario
  or its dimensions (size, duration) are increased. This includes, for example, multi-robot scenarios, assembly tasks or new interaction stations.
\end{itemize}

For a successful implementation either an existing solution has to be increased in robustness or a new approach for an additional task has to be developed. The challenges here lie in the fields of perception, manipulation, navigation and planning.

This years challenges try to introduce the relatively new atwork-commander and the planned set of new objects (see \ref{sec:new_objects}).
These changes seem highly important for the participation of new teams, which is why we encourage every team to participate in atleast one technical challenge.

The exact test definition is still missing at the moment, but as the challenges are not part of the main scoring system, some of the rules may be adjusted to the requirements of the teams.

\begin{comment}
\begin{figure}[h!]
  \centering
  \begin{tikzpicture}
  	\begin{polaraxis}[
        xtick ={0, 90, 180, 270},
        xticklabels= {Manipulation, Perception, Navigation, Planning},
        width=5.5cm,
        height=5.5cm,
        legend pos=outer north east,
    ]
  	\addplot
  		coordinates {(0,1.5) (90,3) (180, 0.5) (270,0.5) (0, 1.5)};  % cp
      \addlegendentry{Cluttered Pick Test}
    \addplot
    	coordinates {(0,3) (90,1.5) (180,1) (270,1) (0,3)};  % pfd
      \addlegendentry{Pick from Drawer Test}
  	\end{polaraxis}
  \end{tikzpicture}
  \label{Examplary Challenge introducing a long term operation based on an extended Final test}
\end{figure}

The challenges of 2021 focussing on perception and manipuation in two scenarios. While "Cluttered Pick Test" (CP) adresses the robustness of perception, the "Pick from Drawer Test" (PFD) is focused on additional complexity by including objects in a drawer. Additonally, the start of a league specific simulator, to ease entry of new teams and enable better scientific evaluation is to be established through Simulation Evaluation Test (SE).

\end{comment}

% !TEX root = ../Rulebook.tex

\newpage
\subsection{Master Communication Test}

The purpose of the \iaterm{Master Communication Test}{MCT} is to introduce the atwork-commander to all (espeically new) teams and encourage them to implement reliable communications.
Therefore, a BTT1 test configuration is created but not sent as whole, but rather as individual transportation tasks.
Robots must confirm a successfully completet task before receiving the next one, meaning they can only perform one task if they do not have logging / feedback implemented.

%% !TEX root = ../Rulebook.tex

\section{Real Object Test}

The purpose of the \iaterm{Real Object Test}{ROT} is to introduce a new set of objects to the league that aims to replace the hard-to-get RoCKin objects.

Therefore, a BTT2 test configuration is created that only includes the new objects defined in section \ref{sec:new_objects}. Teams should be able to participate if they are able to detect and grasp the new objects.

% !TEX root = ../Rulebook.tex

\subsection{Coworker Assembly Test}

In the \iaterm{Coworker Assembly Test}{COT}, 
the goal is for a robot to produce a product while cooperating with a human worker.

Therefore, the robot has to collect three components scattered in the arena and bring them to a single workstation where the human worker waits on the other side of the table.
The robot has to place the objects and then wait for the human to assemble the objects to a product.
Once the human worker (a teammember of the performing team) has assembled the product, he/she must place the product back onto the table and give the robot a completion signal. This may be done via sign (visual) or voice (acoustic). It is not allowed to use controller or keyboard input.

Once the robot has recognized the signal, 
it must detect the assembled product, grasp it and then deliver it to another workstation.
This step concludes the assembly process by imitating the transport of the product to a warehouse.

Three object sets can be assembled during the COT: 
\begin{itemize}
	\item Set 1: Screw M20\_100 and Spacer and Nut M20
	\item Set 2: Bearing2 and  Housing
	\item Set 3: Axis2 and Nut M20  
\end{itemize}

%\newpage
\section{RoboCup Manipulation Challenge}

This challenge is not part of the normal @work technical challenges, but because it's about manipulation, all @Work-Teams are invited to participate. It's a joint challenge between RoboCup@Home and RoboCup@Work on
Autonomous Robot Manipulation (ARM), supported by MathWorks. More details can be found  at \url{arm.robocup.org}.

All teams which are already registered for @Work (or @Home) can participate (or register for the challenge only). All participants will receive a certificate of participation and all teams which delivers a working solution with non-zero performance will receive a \grqq proficiency\grqq \ certificate on robot manipulation and MathWorks tools issued by RoboCup Federation and MathWorks. The winner will receive an award of up to \$5,000 in the form of a grant for research acitivities. This challenge will continue and envolve in the next years.