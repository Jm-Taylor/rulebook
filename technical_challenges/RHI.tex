\newpage
\section{Robot-Human Interaction Test}

\subsection{Purpose and Focus of the Test}
The collection of freely available objects lying on a manipulation zone is the core capability of \RCAW-robots. The \iaterm{Robot Human Interaction Test }{RHI} includes not only a new and realistic objects set to grasp but also includes human workers into the scenario. In this way, the challenge extends the standard Basic Transportation Test~\ref{sec:Basic Transportation Test} where the robot has transport only basic objects without any further interaction. 

\subsection{Scenario Environment}
Basically the BTT2 scenario will be used but with the objects defined in section~\ref{sec:new_objects}. This includes all basic manipulation objects from Table \ref{manipulation_objects} and future objects from Table~\ref{tab:new_objects1} and Table~\ref{tab:new_objects2}.

\subsection{Task}
The task is the same as for a BTT2, with the modification that rocking objects will be replaced by the objects from section~\ref{sec:new_objects}. Furthermore a human worker has to be selected which has to assemble suitable parts together. The human worker has to stand outside the arena and has to be able to collect items from goal workspaces. Additional tools, screws, nuts and washers needed can be carried and prepared by the human worker.
Objects which can be assemble: 
\begin{itemize}
	\item Screw M20\_100 and Spacer and Nut M20
	\item Bearing2 and  Housing
	\item Axis2 and Nut M20  
\end{itemize}

After successfully reassembling of the objects the human worker has to place the objects on the table and the robot has to recognize the new assembled objects and has to transport those to a final goal destination. 


\subsection{Rules}
The following rules have to be obeyed:

\begin{itemize}
	\item A single robot is used.
	\item Six objects have to be picked.
	\item There must be at least 3 decoy objects which must not be picked.
	\item The robot has to start from outside the arena and to stop in the goal area.
	\item A manipulation object counts as successfully grasped as specified in Section~\ref{ssec:GraspingObjects}.
	\item The run is over when the robot reached the final place and the human worker successfully assembled the components or the designated time has expired.
	\item The order in which the teams have to perform will be determined by a draw.
	\item At the beginning of a team's period, the team will get the task specification.
\end{itemize}

\subsection{Scoring}
\begin{itemize}
	\item 200 points are awarded for each correctly and successfully picked object
	\item 125 points are awarded for each correctly placed object.
	\item 50 points are awarded for successfully assembled components.
	\item 250 points are awarded for each correctly and successfully picked assembled object. 
	\item Standard scoring applies for all other aspects
\end{itemize}
