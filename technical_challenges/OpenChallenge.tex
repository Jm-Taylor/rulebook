% !TEX root = ../Rulebook.tex


\newpage
\section{Open Challenge}
During the Open Challenge teams are encouraged to demonstrate recent research results and the best of the robots’ abilities. It focuses on the demonstration of new approaches and applications to industrial tasks.

\subsection{Task}

The Open Challenge consists of a demonstration and an interview part. The performance of the teams is evaluated by a jury consisting of all team leaders, TC and OC members. Although it is supposed to be an open demonstration and the teams shall have freedom to develop their own ideas, a scientific and industrial tie will be considered in the jury's assessment. The topic of the demonstration should focus on one or multiple fields of the RoboCup@Work league namely robot manipulation, robot navigation and mapping, object detection and recognition.

\begin{itemize}
\item[1.] Setup and demonstration: The team has a maximum of 7 minutes for setup, presentation and demonstration.
\item[2.] Interview: After the demonstration, there is another 3 minutes where the team answers questions by the jury members.
\end{itemize}

\subsection{Presentation}
During the demonstration, the team can present the addressed problem and the demonstrated approach.
A video projector or screen, if available, may be used to present a brief (max. 1 minute) introduction to what will be shown. The team can also visualize robot’s internals, e.g., percepts.

\subsection{Jury​ ​evaluation}
Jury​:​ All teams have to provide one person (preferably the team-leader) to follow and evaluate the entire Open Challenge. A jury member is not allowed to evaluate and give points for the own team. The present members of the TC and OC will join the jury.  \par 
Evaluation:​ Both the demonstration of the robot(s), and the answers of the team in the interview part are evaluated. \par
For each of the following evaluation criteria, a maximum number of points (as defined by equation \ref{eq:criteriapoints}) is given per jury member:
\begin{itemize}
\item Overall demonstration
\item Robot autonomy in the demonstration
\item Realism and usefulness for industrial like applications (Can this be ported on real industrial scenarios?)
\item Novelty and (scientific) contribution
\item Difficulty and success of the demonstration
\end{itemize}

The maximum points for each criterion is calculated as:
\begin{equation}\label{eq:criteriapoints}
S_{c,max}=\frac{MaximumPointsForThisChallenge}{NumberOfEvaleuationCriteria}
\end{equation}
The maximum points for this challenge is 200 in \RCAW \YEAR .

Normalization​ ​and​ ​outliers:
\begin{itemize}
%\item The points given by each jury member are scaled to obtain a maximum point count as defined by equation \ref{}
\item The total score for each team is the mean of the jury member scores. To neglect outliers, the N best and worst scores are left out:
\end{itemize}

\begin{equation}\label{eq:totalpoints}
S=\frac{\Sigma TeamLeaderScore}{NumberOfTeams−(2N+1)}
\end{equation} 
$N = 2$ if $NumberOfTeams \geq 10$, $1$ if $NumberOfTeams < 10$. \par


%\subsection{Team-team​ ​interaction}
%There might be a team-team interaction component in the assessment of the open challenge in future years. 

%\subsection{Inter-league​ ​collaboration}
%There might be a inter-league collaboration component in the assessment of the open challenge in future years. 
