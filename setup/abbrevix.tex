%% %%%%%%%%%%%%%%%%%%%%%%%%%%%%%%%%%%%%%%%%%%%%%%%%%%%%%%%%%%%%%%%%%%%%%%%%%%%
%%
%%          $Id: abbrevix.tex 373 2013-02-12 20:32:49Z holz $
%%    author(s): RoboCupAtWork Technical Committee(s)
%%  description: Abbreviations for the RoboCupAtWork RuleBook
%%
%% %%%%%%%%%%%%%%%%%%%%%%%%%%%%%%%%%%%%%%%%%%%%%%%%%%%%%%%%%%%%%%%%%%%%%%%%%%%

%% temp-variables %%%%%%%%%
\newlength{\adxtemp}
\newlength{\adxspace}
\setlength{\adxspace}{3cm}

%% %%%%%%%%%%%%%%%%%%%%%%%%%%%%%%%%%%%%%%%
%%  Commands to use this
%% %%%%%%%%%%%%%%%%%%%%%%%%%%%%%%%%%%%%%%%

%% \abb{abk}
%% to reference an abbreviation
\newcommand{\abb}[1]{\hypertarget{#1}{#1}}

%% \term{the term}
%% print 'the term' in italic,
%% do not include 'the term' in index
%\newcommand{\term}[1]{#1\index{#1}}
\newcommand{\term}[1]{\textit{#1}}

%% \iterm{the term}
%% print 'the term' in italic,
%% include 'the term' in index
\newcommand{\iterm}[1]{\textit{#1}\index{#1}}

%% \nterm{the term}
%% do NOT print 'the term' but include 'the term' in index
\newcommand{\nterm}[1]{\index{#1}}

%% \TERM
%% print first term, add second term to index
\newcommand{\Term}[2]{\textit{#1}\index{#2}}


%% \aterm{the term}{abb}
%% print 'term', 
%% include 'term' with abbreviation 'abb' in abbreviation-index
\newcommand{\aterm}[2]{\settowidth{\adxtemp}{#2}{\hypertarget{#2}{#1 (#2)}\abbex{#2\hspace{\adxspace}\hspace{-\the\adxtemp}#1}}}

%% \iaterm{the term}{abbreviation} 
%% print 'term', 
%% include 'the term' in index, 
%% and include abbreviation in abbreviation-index
\newcommand{\iaterm}[2]{\settowidth{\adxtemp}{#2}{\hypertarget{#1}{\textit{#1} (#2)}\index{#1}\abbex{#2\hspace{\adxspace}\hspace{-\the\adxtemp}#1}}}

%% %%%%%%%%%%%%%%%%%%%%%%%%%%%%%%%%%%%%%%%
%%  Main Abbreviation Definitions
%% %%%%%%%%%%%%%%%%%%%%%%%%%%%%%%%%%%%%%%%

\makeatletter
\newlength{\QZ@TOChdent}% Used to define hanging indents.
\setlength{\QZ@TOChdent}{1.0em}%
\renewcommand{\@pnumwidth}{1.55em}%
\makeatother
%
%\newenvironment{simpleenv}[4]{\clearpage}{\clearpage}%
%\newenvironment{simpleenv}[4]{\clearpage}{\pagebreak}%
\newenvironment{simpleenv}[4]{\pagebreak}{\pagebreak}%
\newcommand{\BaseDiff}{0}%
%\newcommand{\RSpnum}[1]{\makebox[\@pnumwidth][r]{#1}}%
\newcommand{\RSpnum}[1]{\makebox[1.57em][r]{#1}}%
\newcommand{\RSnopnum}[1]{\makebox[\@pnumwidth][r]{}}%
\newcommand{\RSpset}[1]{\RSpnum{#1}}%
%
\newcommand{\printabbex}[4][\jobname]{%
  \typeout{ ***** printabbex #1 #2 #3 #4 for file \jobname.tex}%
  \IfFileExists{#1.and}{%
     \begin{simpleenv}{#1}{#2}{#3}{#4}%
       \pagestyle{plain} %
       \addcontentsline{toc}{chapter}{#2}%
       \chapter*{\textbf{#3}}{#4}%
       \input{#1.and}%
       \vfill%
     \end{simpleenv}%
   }%
   {\typeout{ ***** ERROR: No file #1.and found for file \jobname.tex.}}}%
\renewcommand{\see}[2]{#2 \emph{see} #1}%
\makeatletter
\newcommand{\makeabbex}[1][\jobname]{\begingroup
  \makeatletter
  \if@filesw \expandafter\newwrite\csname #1@adxfile\endcsname
  \expandafter\immediate\openout \csname #1@adxfile\endcsname #1.adx\relax
  \typeout{ ***** Writing abbex file #1.adx for file \jobname.tex.}%
  \fi \endgroup}
%\makeatother
%\makeatletter
\@onlypreamble{\makeindex}%
\newcommand{\abbex}[2][\jobname]{\@bsphack\begingroup
               \def\protect##1{\string##1\space}\@sanitize
               \@wrabbex{#1}{#2}}
\newcommand{\@wrabbex}[2]{\let\thepage\relax
   \xdef\@gtempa{\@ifundefined{#1@adxfile}{}{\expandafter
      \write\csname #1@adxfile\endcsname{\string
      \indexentry{#2|}{\thepage}}}}\endgroup\@gtempa
   \if@nobreak \ifvmode\nobreak\fi\fi\@esphack}
\makeatother
\newcommand{\indxspace}{\par\vspace{\BaseDiff\baselineskip}}
\makeatletter
\newcommand{\IndexSet}{%
\renewcommand{\@idxitem}{\par\setlength{\leftskip}{0pt}%
                         \setlength{\hangindent}{\QZ@TOChdent}}%
\renewcommand{\subitem}{\par\setlength{\leftskip}{0.25in}%
                         \setlength{\hangindent}{\QZ@TOChdent}}%
\renewcommand{\subsubitem}{\par\setlength{\leftskip}{0.5in}%
                         \setlength{\hangindent}{\QZ@TOChdent}}%
\renewcommand{\indexspace}{}
\renewcommand{\indxspace}{\par\vspace{\BaseDiff\baselineskip}}
\renewenvironment{theindex}{%
                \setlength{\parindent}{\z@}%
                \parskip\z@ \@plus .3\p@\relax
                \setlength{\rightskip}{\@tocrmarg}%
                \setlength{\parfillskip}{-\rightskip}%
                \let\item\@idxitem}
} %% End of the IndexSet definition
\makeatother

\newcommand{\printidx}{\addcontentsline{toc}{chapter}{Index}\printindex}
\newcommand{\printabx}{\printabbex{Abbreviations}{Abbreviations}{}}
%\newcommand{\printabbrevidx}{\printindex[abb]{Abk�rzungen}{Abk�rzungen}{}}
\newcommand{\printabbrevidx}{\printabbex{Abbreviations}{Abbreviations}{}}
%\newcommand{\makeabbrevidx}{\makeindex[abb]}
\newcommand{\makeabbrevidx}{\makeabbex}

% Local Variables:
% TeX-master: "../Rulebook"
% End:
