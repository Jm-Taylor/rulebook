% !TEX root = ../Rulebook.tex

\section{Advanced Transportation Test}
\label{sec:Advanced Transportation Test}

The \iaterm{Advanced Transportation Test}{ATT} has its origins as a BTT, but greatly increases the difficulty of the tasks to perform by introducing more random elements. 

This includes the introduction of arbitrary surfaces, shelfs and precise placement tables, barriertape as visual obstacles and the containers as target objects. With the object count further increasing, task optimization and replanning in case of failure also becomes more relevant.

As with the BTTs, there are two versions of the ATT, which gradually introduce the more challenging elements of \RCAW.
The following paragraphs summarize the two different levels but DO NOT override the test specification in table \ref{tab:Instances}.

\paragraph{ATT1}
\begin{itemize}
\item Six randomly selected objects have to be transported.
\item Three randomly selected decoy objects are placed onto one or more randomly
\item There will be five active service areas (four tables and one shelf)
\item All table heights are used (0-15 $\si{\centi\meter}$).
\item Two Objects must be placed on a shelf (top part).
\item Virtual Obstacles (Barriertapes) are placed inside the arena (one blocking, one non-blocking).
\item Two service areas will have an arbitrary surface.
\end{itemize}

\paragraph{ATT2}
\begin{itemize}
\item Seven randomly selected objects have to be transported.
\item Five randomly selected decoy objects are placed onto one or more randomly
\item There will be six active service areas (four tables,  one shelf and one Precise Placement table)
\item All table heights are used (0-15 $\si{\centi\meter}$).
\item One object must be picked from a shelf (lower part).
\item One object must be Precise Placed.
\item Two objects must be placed into a container (preferably one to each color).
\item One visual and one physical obstacle is placed inside the arena (both semi-blocking).
\item Three service areas will have an arbitrary surface.
\end{itemize}