% !TEX root = ../Rulebook.tex

\newpage
\section{Basic Manipulation Test}
\label{sec:Basic Manipulation Test}

\paragraph{Purpose and Focus of the Test}
The purpose of the \iaterm{Basic Manipulation Test}{BMT} is to demonstrate basic manipulation capabilities by the robots, like grasping, turning, or placing an object.
\par
The focus is on the manipulation and on demonstrating safe and robust grasping and placing of objects of different size and shape. Therefore, the number of service areas will be constraint to two, one source area and one target area, which are close to each other. 

\paragraph{Scenario Environment}
Additionally to environmental elements, different manipulatable objects will be placed on the specified service areas.

\paragraph{Manipulation Objects}
The manipulation objects used in this test are defined by the instances described in Table~\ref{tab:Instances}.

\paragraph{Task}
The task consists of a sequence of grasp and place operations, with a small base movement in between. The objective is to move a set of objects from one service area into another. To complete the task the source and the target destination have to be reached at least once.
\par
The task specification consists of:
\begin{itemize}
	\item[--] The specification of the initial place
	\item[--] A source location, given as place (any one)
	\item[--] A destination location, given as place (any one, but nearby the source location)
	\item[--] A list of objects to manipulated from the source to the destination service area
	\item[--] The specification of a final place for the robot
\end{itemize}


\paragraph{Rules}
The following rules have to be obeyed:

\begin{itemize}
\item The order in which the teams have to perform will be determined by a draw.
\item The robot will get the task specification from the referee box.
\item A service area counts as successfully reached as defined in Section~\ref{ssec:Navigating}
\item A manipulation object counts as successfully grasped as defined in Section~\ref{ssec:GraspingObjects}
\item A manipulation object counts as successfully placed, if the robot has placed the object into the correct destination service area as described in Section~\ref{ssec:PlacingObjects}.
\item  The run is over when the robot reached the final place or the designated time has expired.
\item The score for this test will be calculated as defined in \ref{sec:ScoringAndRanking}.
\end{itemize}


