% !TEX root = ../Rulebook.tex
\newpage
\section{Basic Manipulation Test}

\paragraph{Purpose and Focus of the Test}
The purpose of the \iaterm{Basic Manipulation Test}{BMT} is to demonstrate basic manipulation capabilities by the robots, like grasping, turning, or placing an object.
\par
The focus is on the manipulation and on demonstrating safe and robust grasping and placing of objects of different size and shape. Therefore, only minimal movement of the robot is required.
\par
Some minor movement is intentionally designed into this test in order to force the teams to perform dynamic assessment of the situation (e.g. estimating positions of manipulation objects, determining grasp positions, etc.) and to avoid that solutions depending on completely known initial situations and well-calibrated systems are possible.

\paragraph{Scenario Environment}
The arena used for this test contains basically all elements as for the Basic Navigation Test. Additionally to environmental elements (walls, service areas, floor markers, etc.), different manipulatable objects will be placed on the service areas.

\paragraph{Manipulation Objects}
The manipulation objects used in this test are defined by the instances described in Table~\ref{tab:Instances}.

\paragraph{Task}
A single robot is used. The robot can be placed in an arbitrary starting location by the team. The task consists of a sequence of grasp and place operations, with a small base movement in between. The objective is to move a set of objects from one service area into another. The task is finished once all objects are moved or when the time foreseen for the run ends.
\par
The task specification consists of:
\begin{itemize}
	\item The specification of the initial place (e.g. \texttt{D0}, \texttt{S5}, \texttt{U2})
	\item A source location, given as place (any one)
	\item A destination location, given as place (any one, but nearby the source location)
	\item A list of objects to manipulated from the source to the destination service area
	\item The specification of a final place for the robot (which does not need to be reached)
\end{itemize}

%
% \subsection{Complexity Options}
%
% \subsubsection{Manipulation Object complexity (pick one):}
%
% \begin{itemize}
% \item Choose objects (bonus factor =  0.0): The team can freely choose the manipulation objects.
% \item Few objects (bonus factor =  0.2): The team can freely choose a set of five manipulation objects.
% \item All objects (bonus factor =  0.5): All manipulations objects can be used.
% \end{itemize}
%
%
% \subsubsection{Decoy Object complexity (pick one):}
%
% \begin{itemize}
% \item No decoy objects (bonus factor =  0.0): No decoy objects will be used.
% \item Few decoy objects (bonus factor =  0.2): The team can freely choose a set of three manipulation objects, to be used as decoy.
% \item All decoy objects (bonus factor =  0.3): All manipulations objects can be used as decoy.
% \end{itemize}
%
%
% \subsubsection{Orientation Complexity (bonus factor = 0.2):}
% The manipulation objects can be placed in all orientations.
%
% \subsubsection{Rotation Complexity (bonus factor = 0.2):}
% The manipulation objects can be placed in all orientations.
%
% \subsubsection{Position Complexity (bonus factor = 0.2):}
% The manipulation objects while be placed by the referees.
%
 \paragraph{Rules}
 The following rules have to be obeyed:

 \begin{itemize}
 \item The order in which the teams have to perform will be determined by a draw.
 \item The team can setup the robot anywhere inside the arena.
 \item The robot will get the task specification from the referee box.
 \item A manipulation object counts as successfully grasped as defined in Section~\ref{ssec:GraspingObjects}
 \item A manipulation object counts as successfully placed, if the robot has placed the object into the correct destination service area as described in Section~\ref{ssec:PlacingObjects}.
 \item The time is stopped when the robot has completed the task by placing the last object of the task specification. If a team cannot complete the task within the designated time, the run will be stopped.
 \end{itemize}

%
%
% \subsection{Scoring}
% Points are awarded as follows:
%
% \begin{itemize}
% \item 100 points are awarded for successfully grasping a manipulation object that is part of the task specification..
% \item 50 points are awarded for successfully placing a manipulation object into the destination service area.
% \item - 100 points for grasping a wrong manipulation object.
% \item 50 points are awarded for completing the task specification completely correct.
% \item The reached points of a test will be multiplied with a defined complexity factor depending on the previously chosen complexity level.
% \end{itemize}
