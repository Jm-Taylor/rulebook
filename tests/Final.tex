\section{Final (Example)}

\subsection{Purpose and Focus of the Final}
The purpose of the Final is to show what each robot can do in combination. The Final test will be a combination of the other tests. To show how the final test could look like the following is added:

\subsection{Example Scenario}

Basically the scenario is a BTT with a PPT and Navigation Complexity included, the following adjustments have been made:

\begin{itemize}
\item Time limit is 10 min. 
\item 5 Objects will have to be transported in total. 
\item Two objects will be used twice. 
\item 3 Service- and 3 Destination-areas will be used. One dest. area is a PPT area.
\item The robot will have to start at the entrance, and to complete the run exit through the exit.  
\end{itemize}


\subsection{Complexity}
	
\begin{itemize}
\item Obstacle complexity (BNT)
\item Barrier Tape complexity (BNT)
\item Manipulation object complexity (BMT)
\item Decoy object complexity (BMT)
\item Orientation complexity (BMT)
\item Rotation complexity (BMT)
\item Position complexity (BMT)
\item Speed Complexity (CBT)
\item PPT Orientation Complexity (PPT)
\item PPT Rotation Complexity(PPT)
\end{itemize}

		
\subsection{Scoring}

\begin{itemize}
\item 50 points for (the first time) reaching a service or destination area. An area counts as reached when any part of the robot is within 1m of the area and the robot is oriented towards the area. The robot does not have to stand still. 
\item 50 points for grasping a (correct) object. Grasping is defined like in BTT. 
\item 50 points for transporting and putting to the correct destination area. 
\item 50 points for placing into the correct cavity of a PPT plate. 
\item If an object is placed into the wrong cavity or to the wrong area no penalty points apply. 
\item 50 points for completing the test, + time bonus points according to rules.
\item Other penalty points like losing objects etc. apply according to the BTT test, if not stated otherwise above.
\end{itemize}

.

