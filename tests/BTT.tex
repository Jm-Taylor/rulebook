\section{Basic Transportation Test (BTT)}

\subsection{Purpose and Focus of the Test}
The purpose of the Basic Transportation Test is to assess the ability of the robots for combined navigation and manipulation tasks. 
The focus of the Basic Transportation Test is to assess the ability of a team to deal with flexible task specifications, esp. concerning information about object constellations in the source and the target locations, and task constraints such as limits on the number of objects allowed to carry simultaneously, etc.  

\subsection{Scenario Environment}
The arena used for this test contains all elements as for the Basic Manipulation Test. 

\subsection{Manipulation Objects}
The manipulation objects in this test include the objects specified in Section 5.5.1.

\subsection{Task}
A single robot is used, which is initially positioned outside of the arena near a gate to the arena. The task is to get several objects from the source service areas (such as S1, T7, or U3), and to deliver them to the destination service areas (e.g. D1 and D3). Robots may carry up to three objects simultaneously. 
\par
The task specification consists of two lists:
The first list contains for each service area a list of manipulation object descriptions. The descriptions are similar as those used for the Basic Manipulation Test. 
The second list contains for each destination service area a configuration of manipulation objects the robot is supposed to achieve. The configuration specification is similar as used in the Basic Manipulation Test. 

Two example task specifications are given here:
\begin{itemize}
\item BTT\textless initialsituation(\textless S6,line(M20\_100,F20\_20\_B)\textgreater \textless S7,line(V20)\textgreater);goalsituation(\textless S1,line(M20\_100,F20\_20\_B)\textgreater \textless S2,line(V20)\textgreater )\textgreater
\item  BTT\textless initialsituation(\textless S6,line(S40\_40\_B,F20\_20\_B)\textgreater \textless S7,line(M20\_100)\textgreater) ;goalsituation(\textless S1,line(F20\_20\_B,F20\_20\_B)>\textless S2,line(M20\_100)\textgreater )\textgreater

\end{itemize}

The term “line” in the task specification can be ignored.


\subsection{Complexity Options}
The same complexity scoring as in BMT applies.

\subsection{Rules}
The following rules have to be obeyed:

\begin{itemize}
\item The order in which the teams have to perform will be determined by a draw.
\item A team has a time period of 10 minutes to complete a run.
\item At the beginning of a team’s period, the team will get the task specification. The robot has to start from outside the arena.
\item After the team’s robot starts, it must move into the arena and attempt to complete the task. 
\item A manipulation object counts as successfully grasped as specified in Section 5.5.3. 
\item It is not allowed to place manipulation objects anywhere except for the robot itself and any of the available service areas.
\item A robot may carry up to three objects at the same time.
\item The time is stopped when the robot has completed the task (delivered all objects to the right locations and left the arena through the exit gates). If a team cannot complete the task within 10 minutes, the run will be stopped after 10 minutes. 
\item The team must leave the arena within a minute after completing the task.
\end{itemize}



\subsection{Scoring}
Points are awarded as follows:

\begin{itemize}
\item 75 points are awarded for successfully grasping a manipulation object required in the task specification.. 
\item - 75 points if a wrong object has been grasped
\item 75 points are awarded for successfully placing a manipulation object into the destination service area.
\item 50 points are awarded for completing the task specification completely correct. 
\item The reached points of a test will be multiplied with a defined complexity factor depending on the previously chosen complexity level
\end{itemize}

