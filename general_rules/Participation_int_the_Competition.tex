\section{Participation in the Competition}\label{sec:participation_in_the_competition}
Participation in \RCAW requires successfully passing a qualification procedure. This procedure is to ensure the quality of the competition event and the safety of participants. Depending on constraints imposed by a particular site or the number of teams interested to participate, it may not be possible to admit all interested teams to the competition.

%\todo[inline]{Needs to be reviewed by Execs}

\subsection{Steps to Participate}
All teams that intend to participate at the competition have to perform the following steps:

\begin{enumerate}
	\item Preregistration (may be optional; currently by sending email to the TC)
	\item Submission of qualification material, including a team description paper, a promotional videos and possibly additional material like designs or drawings
	\item Final registration (qualified teams only)
  \item Entering the competition
\end{enumerate}


All dates and concrete procedures will be communicated in due time in advance.

\subsection{Qualification}
The qualification process serves a dual purpose: It should allow the TC to assess the safety of the robots a team intents to bring to a competition, and it should allow to rank teams according to a set of evaluation criteria in order to select the most promising teams for a competition, if not all interested teams can be permitted. The TC will select the qualified teams according to the qualification material provided by the teams. The evaluation criteria will include:

\begin{itemize}

	\item Team description paper
	\item Relevant scientific contribution/publications
	\item Professional quality of robot and software
	\item Novelty of approach
	\item Relevance to industry
	\item Performance in previous competitions
	\item Contribution to \RCAW league, e.g. by
		\begin{itemize}
			\item Organization of events
			\item Provision and sharing of knowledge
		\end{itemize}
	\item Team promo video
	\item Team web site

\end{itemize}

\subsection{Team Description Paper}
The \iaterm{Team Description Paper}{TDP} is a central element of the qualification process and has to be provided by each team as part of the qualification process. All TDPs will be included in the CD proceedings of the \RC Symposium.
The TDP should at least contain the following information in the author/title section of the paper:

\begin{itemize}
	\item Name of the team (title)
	\item Team members (authors), including the team leader
	\item Link to the team web site
	\item Contact information
\end{itemize}


The body of the TDP should contain information on the following:

\begin{itemize}
	\item focus of research/research interest
	\item description of the hardware, including an image of the robot(s)
	\item description of the safety systems used on the robot, including emergency stop procedure
	\item description of the software, esp. the functional and software architectures
	\item innovative technology (if any)
	\item reusability of the system or parts thereof
	\item applicability and relevance to industrial tasks
\end{itemize}

The team description paper should cover in detail the technical and scientific approach, while the team web site should be designed for a broader audience. Both the web site and the TDP have to be written in English. Alongside the TDP, the TC will - starting 2019 - also require a video file presenting the robot, see Section~\ref{ssec:promotional_video}. All TDPs must be written using the following template: \href{https://www.overleaf.com/latex/templates/springer-lecture-notes-in-computer-science/kzwwpvhwnvfj#.WtR5Hy5ua71}{Overleaf TDP template} 

\subsection{Promotional Video}
\label{ssec:promotional_video}
In order to better judge the quality of a team's qualification, the TC asks every team, established or new, to submit a video file describing the robot and its design. The video should clearly demonstrate the robot's ability to perform the tasks required in the challenge, such as autonomous navigation, picking, and placing. Desired elements include visualizing the sensory capabilities of the robot, i.e., seeing what the robot sees, and the plan currently followed by the robot. Spoken language/an audio stream is not required. Ideal video resolution is 1080p with a 16:9 ratio. For large files, please provide a download link.
This file will also be played as explanatory and promotional material during the competiton.

\subsection{Entering the Competition}%
\label{sec:participation}
To actually participate in the competition, teams need to provide at least one set of all objects used in the
competition. This includes the Basic and Advanced Object Set, see Section~\ref{subsec:containers} and the
containers, see Section~\ref{ssec:ManipulationObjects}. The objects are used as part of the league's object pool, which are
used to realize the official tasks generated by the Atwork Commmander, see Section~\ref{sec:atwork-commander}. Provision
of arbitrary surfaces, see Section~\ref{subsec:Arbitrary_Surfaces_and_Decoys} is optional, but highly appreciated to ensure a large variarity of
surfaces for the tasks. If teams want to make use of existing simplifications, they also need to provide the necessary
equipments for these. For more informations refer to Sections~\ref{ssec: April Tagged Object Set}~and~\ref{ssec:simplification_ato}.
