\section{Introduction and Motivation}

In order to foster the development of new teams and to increase cooperation among established teams, the league announces an Open Source Award. As demonstrated
in other \RC major leagues, releasing software and/or hardware as open source fosters the overall progress of the league. Similarly, to other open source awards in \RC, all institutions, persons and teams who took part in national and international \RCAW competitions in \YEAR or the previous year are eligible to participate.

\section{Application}
The application should contain the following items:

\begin{itemize}
	\item A technical report and description (max. 8 pages in Springer LNCS style) about the open source artifact (software/hardware or both). The report should briefly describe the open source project objectives, design decisions and most importantly should exemplify it's importance for the \RCAW competition and community.
	\item A online reference, documentation, tutorial (e.g. website, Github page etc.) for the presented open source material. This includes also statements about licensing (e.g. which kind of open source license such as GPL, LGPL, Apache, etc.) and usage.
\end{itemize}

Application deadline is the: 01.05.2016. Application material needs to be send via email to: \texttt{rc-work-tc@lists.robocup.org}. Please note, in case the evaluation committee receives  only applications which do not fulfill the desired level of quality, the award will not be given in \YEAR. The winner(s) will be announced during the RoboCup \YEAR award ceremony.


\section{Evaluation}
Evaluation is performed by an external jury chosen by the EC and TC. The evaluation criteria are the following:

\begin{itemize}
	\item Relevance: \emph{Is the presented material relevant for \RCAW? Can the material be applied in the context of \RCAW?}
	\item Originality: \emph{Does the presented material solve a problem/issue in \RCAW in a very appealing, general approach?}
	\item Technical Quality: \emph{Is the presented material well-designed and well-developed. This also includes coding style etc.? }
	\item Presentation: \emph{Is the presentation of the material appealing and complete for \RCAW purpose?}
\end{itemize}
