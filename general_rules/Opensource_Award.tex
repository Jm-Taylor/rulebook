\section{Introduction and Motivation}
In order to foster new teams development and to increase cooperation among established teams, the league announces an Open Source award. As demonstrated
in other \RC major leagues releasing software and/or hardware as open source fosters the overall progress of the league. Similarly, to other Open Source awards in \RC, all \RCAW institutions, persons and teams who took part in national and international \RCAW competitions in 2015 and 2016 are eligible to participate. 

\section{Application}
The application should contain the following items: 

\begin{itemize}
	\item A technical report and description (max. 8 pages in Springer LNCS style) about the open source artifact (software/hardware or both). The report should briefly describe the open source project objectives, design decisions and most importantly should exemplify it's importance for the \RCAW competition and community.   
	\item A online reference and eventually documentation, tutorial (e.g. website, Github page etc.) for the presented open source material. This includes also statements about licensing (e.g. which kind of open source license such as GPL, LGPL or Apache to name a few) and usage.  
\end{itemize}

Application deadline is the: XX-XX-XX. Application material needs to be send via e-Mail to: XXXXX. Please note, in case the evaluation committee receives less than three applications the award will not be given in 2016. 

\section{Evaluation}
Evaluation is performed by an evaluation committee which consists of the following members:

\begin{itemize}
	\item Walter Nowak (\RCAW Exec.)
	\item Nico Hochgeschwender (\RCAW Exec.)
	\item MORE TBD
\end{itemize}

The evaluation criteria are the following:\todo{add specific dates, and evaluation committee}

\begin{itemize}
	\item Relevance: \emph{Is the presented material relevant for \RCAW? Can the material be applied in the context of \RCAW?}
	\item Originality: \emph{Does the presented material solve a problem/issue in \RCAW in a very appealing, general approach?} 
	\item Technical Quality: \emph{Is the presented material well-designed and well-developed. This also includes coding style etc.? }
	\item Presentation: \emph{Is the presentation of the material appealing and complete?}
\end{itemize}

