% !TEX root = ../Rulebook.tex

\chapter{Summary of Changes}

\begin{comment}
This chapter provides an overview for experienced teams that know the rules and just need an update on what is new for the specific year. 
All new teams are strongly advised to read the whole rule book thoroughly.
\end{comment}


\section{General Changes for 2022}

In 2021, the first virtual robocup was held online via discord, zoom and youtube live.
As a lot of new teams came into the league which never experienced a in-person robocup,
it came clear that the rulebook was missing out on specific information and rule definitions.
A lot of the rules have been habit and spoken agreements during the years,
which included crucial elements such as handling of robot collisions, repeating runs,
on-site competition organization and more.

This years rulebook tries to clarify all these rules, while also releasing some constraints 
while enforcing more robot safety. In the following, 
some rule summaries are being made about the following chapters.
Please carefully read the paragraphs anyways.

\section{Chapter 2: League Organization}

Discord has proven to be a very useful tool to organize 
robocups as it allows teams to communicate. All participating teams should join our Discord to keep up on news and announcements.

\section{Chapter 3: Robot Rules}

The size constraints were relaxed to allow more chassis types.
However, certain safety requirements must be met to participate in the competition.

\section{Chapter 3: Environment Specification}

The requirements for a robot to autonomously navigate the arena have been specified. Robots must fit those specs to participate.

The arena elements and their role in the competition are defined, declaring a task location a service area.
Robots must perform various manipulation tasks at service areas, mostly by handling a set of objects.



\section{Chapter 4: The Competition}

The on-site process is defined and explained, most importantly the rules and schedule of runs. Robots must autonomously perform a set of tasks at different service areas, where the exact task definition is defined by so called tests.

Each team has one performance slot for each test type (currently 7+1), with the option to repeat a lower-scoring test once in a later timeslot if possible.
A performance slot consists of a prep phase, a run phase and the end phase, where the performing team prepares and executes their run and then gets their performance then evaluated by the refs.

\section{Chapter 5: Test Definition}

Clarification of the basic manipulation, basic transportation, precise placement and rotating table test.


\section{Chapter 6: Scoring Adjustments}

The scoring for successful task execution and errors has been updated to ease the competition for newer teams while keeping the ambitions for more difficult tasks with boni.

\section{Chapter 7: Virtual Cups}

Taking the rules from 2021s virtual cup, 
requirements for arena setups at home are defined to allow teams to participate in their own laboratory.
They must livestream their test performance with a professional camera setup to allow refs worldwide to evaluate the performance.

\section{Chapter 8: Technical Challenges}

Three new technical challenges have been introduced to help evolve the league and the scientific challenges. 
The exact specification of the individual tests is yet to be made.
	

\begin{comment}

\martin{Changes: 
	\begin{itemize}
		\item Added siunix-package in preamble
		\item Added caption option in todonote in preamble
		\item Added Rotating Table description, and drawing
		\item Updated Shelf description, and drawing
		\item Added some labels
		\item Added Final in tests chapter (was already written)
		\item Containers descriptions and example, moved to manipulation
		\item object description and printed rocking objects
		\item new objects (2 Lists)
		\item shopping list with links visited January 2022
		\item precision placement table in environment
		\item safty check procedure
		\item technical challeng, Robot Human Interaction with new objects
		\item technical challeng, new objects
		\item aerial robots forbidden
		\item not stop in tdp
		
	\end{itemize}
}

\leander{Changes: 
	\begin{itemize}
		\item \ref{subsubsec: Markup Tape} Markup Tape: introduce green electrical tape
		\item \ref{subsubsec: Workstations} Workstations: 0cm workstations are not marked with blue/white tape anymore
		\item \ref{subsec: Walls and virtual Walls} Walls and virtual Walls: clarifications
		\item \ref{subsubsec: Start and Goal Area} Start and Goal Area: add chapter 
	\end{itemize}
}

\marco{Changes: 
	\begin{itemize}
		\item Arena description
		\item Parc fermé rules
		\item Detailed competition description
		\item POINTS, PENALTIES, BONUS, COLLISIONS
	\end{itemize}
}



\section{Adjustments for the virtual RoboCups}

Due to the Covid-19 pandemic, the 2020 and 2021 international robocups are cancelled. 
Some of the leagues, including the RoboCup@Work league, will be held online.
As this brings new challenges for the teams (e.g. arena building) and the committees (comparing teams and scoring),
the technical committee decided to set some rules regarding the arena setup, scoring, general participation rules and the Technical Challenges.

These can be found in chapter \ref{cha: VRC}.  


\section{General Changes}
\begin{itemize}
  \item BNT will be excluded from the instance list. According to the adapted scoring of the last year (rewards for reaching correct destinations) a separate run focusing on navigation is not necessary anymore.
  \item The preparation time was increased to 3 minutes (1 minute in 2019).
  \item The classification of a collision (major/minor) is more specific now (see \ref{tab:collisions}).
  \item The Referee Box avoids object distribution patterns, where one manipulation zone is target AND source of a transportation task
  \item The minimum passage width of 80cm was clarified in Section~\ref{sec:ArenaDesign}
  \item The meaning of the different types of tape in the arena was modified. Please review Sections~\ref{sec:ArenaDesign},~\ref{sec:penalties}~and~\ref{sec:Collisions}
\end{itemize}

\section{Robot Requirements}
\begin{itemize}
  \item The size constraints for the robots were removed to allow more versatile robot designs. 
  		However, the arena specification declares 80 cm as the minimum distance between fixed arena elements.
  		Teams with bigger robots will have disadvantages regarding navigation, as they may get stuck in narrow arena passages. 
  		See \ref{ssec:RobotDesignAndConstraints}.
  \item A more precise definition of the emergency (hard) stop was defined.
  \item A cap on the battery capacity of 500Wh was given
  \item A speed cap of 1.5m/s was introduced for safety
\end{itemize}

\section{Team Requirements}
\begin{itemize}
  \item All teams are required to bring appropriate storage equipment(e.g. lipo bags) for their batteries to the competition
  \item All teams must educate their team members in correct battery usage 
\end{itemize}

\section{Scoring}
\begin{itemize}
  \item The penalization of barrier tape was changed to a relative deduction. Barrier tape now induces a 5\% penalty on the final points of the run, up to 20\%.
\end{itemize}

\section{Technical Challenges}
\begin{itemize}
  \item The Cluttered Pick Test was more clarified. The test involves now picking \textbf{and placing} of three objects. Also Points for navigation are not given anymore. 
  \item Another challenge regarding the definition of a uniform simulation environment was added.
\end{itemize}

\end{comment}