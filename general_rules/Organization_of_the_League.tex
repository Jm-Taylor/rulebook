\section{Organization of the League}\label{sec:organisation_of_the_league}

\subsection{League Committees}
The following list of committees is implemented for \RCAW.

\subsubsection{Executive Committee}

\iaterm{Executive Committee}{EC} members are responsible for the long term goals of the league and thus have also contact to other leagues as well as to the \RC Federation. The EC presents the league and its achievements to the \RC Federation every year and gets feedback to organize the league. All EC members are also members of the Technical Committee. EC members are elected by the Board of Trustees and appointed by the President of the \RC Federation, they serve 3-year terms. The current EC members are:

\begin{itemize}
	\item Asadollah Norouzi, \textit{Singapore Polytechnic}
    \item Christoph Steup, \textit{Otto von Guericke University Magdeburg}
\end{itemize}


\subsubsection{Technical Committee}
The \iaterm{Technical Committee}{TC} is responsible for technical issues of the league, most notably the definition of the rules, such as compliance of the robots with rules and safety standards, the qualification of teams, the adherence to the rules as well as the resolution of any conflicts that may arise during competition. The current TC members are:

\begin{itemize}
    \item Marco Masannek, \textit{Nuremberg Institute of Technology}
	\item Lucas Reinhart, \textit{University of Applied Sciences Würzburg-Schweinfurt}
    \item Kenny Voo, \textit{Nanyang Technological University} 
    \item Martin Sereinig, \textit{University of Innsbruck} 
    \item Leander Bartsch, \textit{Otto von Guericke University Magdeburg} 
\end{itemize}


\subsubsection{Organizing Committee}
The \iaterm{Organizing Committee}{OC} is responsible for all aspects concerning the practical implementation of competition, most notably for providing the competition arenas, ensuring their conformity with the rules, and any objects and facilities required to perform the various tests. Further, the Committee is responsible for assigning space to teams in the team area, the organization and scheduling of meetings, the nomination and scheduling of referees, the scheduling and timely execution of tests and stages, recording and publishing competition results, and any other management duties arising before, during, and after a competition. The current OC members are:

\begin{itemize}
    \item Franziska Labitzke, \textit{Otto von Guericke University Magdeburg}
    \item Hauke Petersen, \textit{Otto von Guericke University Magdeburg} 
    \item Sally Zeitler, \textit{Nuremberg Institute of Technology}
    \item Yusuf Pranggonoh, \textit{?}
\end{itemize}


%\subsubsection{Industrial Advisory Board}

%The role of the Industrial Advisory Board is to ensure the industrial relevance of the tests and the overall competition. It allows representatives from industry to voice interesting new problems, which may be included in existing or new test scenarios.
%\par
%Currently, the Industrial Advisory Board has no members yet.

%\subsubsection{ Research Advisory Board}

%The role of the Research Advisory Board is to ensure that the tests and scenarios are innovative and relevant from a research point of view. The members will be recruited from academic institutions and research organizations.
%\par
%Currently, the Research Advisory Board has no members yet.

\subsection{League Infrastructure}
In order to provide a forum for continuous discussions between teams and other stakeholders, the league builds and maintains an infrastructure consisting of a web site, mailing lists, and repositories for documentation, software, and data. The infrastructure is complemented by a minimum infrastructure to be built and maintained by teams, i.e. teams should eventually create their own web page to which the \RCAW League's web pages can be linked.


\subsubsection{Infrastructure Maintained by the League} \label{ssec:LeagueInfrastructure}

\paragraph{Website}
The official website of \RCAW is at
\begin{center}
\url{https://atwork.robocup.org/}.
\end{center}

\paragraph{Discord Server}
The official Discord server of \RCAW is
\begin{center}
	\href{Official Discort Server}{https://discord.gg/z6Yn6UvhxU}
\end{center}


This web site is the central place for information about the league. It contains general introductory information plus links to all other infrastructure components, such as a league wiki, the mailing lists, important documents such as this rule book, announcements of upcoming events as well as past events and participating teams.

\paragraph{Mailing Lists}
The league maintains several mailing lists:

\texttt{rc-work@lists.robocup.org} this is the general \RCAW mailing list. Anyone can subscribe, but a real name must be provided either as part of the email address or being specified on the mailing list subscription page. The list is moderated in order to avoid abuse by spammers. New members can subscribe to this list here: 
	\begin{center}
		\url{https://lists.robocup.org/cgi-bin/mailman/listinfo/rc-work}.
	\end{center}

\texttt{rc-work-tc@lists.robocup.org} this is the mailing list for the TC. Posts from non-members have to be approved by the list moderator. Approvals will be given only in well-justified cases.
\begin{center}
\url{https://lists.robocup.org/cgi-bin/mailman/listinfo/rc-work-tc}
\end{center}



\paragraph{Repositories}
Several repositories are publicly available under the official \RCAW Github account:
\begin{center}
\url{https://github.com/robocup-at-work}
\end{center}

The repositories provide 3D models for the manipulation objects, their corresponding PPT cavities, and all arena elements. Additionally, the sources to this rulebook, the implementation of the referee box, and various tools can be found.


\subsubsection{Infrastructure Maintained by Teams}
Each team is requested to build and maintain a minimum infrastructure for its team. This infrastructure consist of

\begin{itemize}
	\item team web site,
	\item team contact, and
	\item team mailing address.
\end{itemize}

The team web site should contain the following information:

\begin{itemize}
	\item Name of the team, and team logo, if any
	\item Affiliation of the team
	\item Team leader including full contact information
	\item List of team members
	\item Description of the team's research interest and background
	\item Description of specific approach pursued by the team
	\item Description of the robot(s) used by the team
	\item List of relevant publications by team members

\end{itemize}

The team contact should be the official contact of the team. Usually, for university-based teams, this would be an academic person such as a professor or post-doc, who should, however, be responsive and be able to answer quickly when contacted by email.
\par
The team mailing address should be an email alias, which should be used to subscribe the team to the general \RCAW mailing list. The email alias should at least include the team contact and the team leader.
