\section{Organization of the League}\label{sec:organisation_of_the_league}

\subsection{League Committees}
The following list of committees is implemented for RoboCup@Work.

\subsubsection{Executive Committee}

Executive Committee (EC) members are responsible for the long term goals of the league and thus have also contact to other leagues as well as to the RoboCup Federation. The Executive Committee presents the league and its achievements to the RoboCup Federation every year and gets feedback to organize the league. All committee members are also members of the Technical Committee. Executive Committee members are elected by the Board of Trustees and appointed by the President of the RoboCup Federation, they serve 3-year terms. The current EC members are:

\begin{itemize}
	\item Walter Nowak, Locomotec GmbH
	\item Nico Hochgeschwender, Bonn-Rhein-Sieg University
\end{itemize}


\subsubsection{Technical Committee}
The Technical Committee (TC) is responsible for technical issues of the league, most notably the definition of the rules, the qualification of teams, the adherence to the rules as well as the resolution of any conflicts that may arise during tournaments. The current TC members are:

\begin{itemize}
	\item Jan Carstensen, Leibniz Universit\"at Hannover
	\item Sebastian Zug, Otto-von-Guericke-University Magdeburg
	\item Yiyan Wang, Singapore Polytechnic
\end{itemize}


\subsubsection{Organizing Committee}
The Organizing Committee (OC) is responsible for the practical implementation of tournaments, most notably for providing test arenas and any objects and facilities required to perform the various tests, scheduling tests, assigning and managing referees, recording and publishing competition results, and any other management duties arising before, during, and after a tournament. The current OC members are:

\begin{itemize}
	\item Philipp Busse, Otto-von-Guericke-University Magdeburg
	\item Asadollah Norouzi, Singapore Polytechnic
	\item Weiwei Shang, University of Science and Technology of China
\end{itemize}

\subsubsection{Industrial Advisory Board}
The role of the Industrial Advisory Board is to ensure the industrial relevance of the tests and the overall competition. It allows representatives from industry to voice interesting new problems, which may be included in existing or new test scenarios. 
\par
Currently, the Industrial Advisory Board has no members yet.

\subsubsection{ Research Advisory Board}
The role of the Research Advisory Board is to ensure that the tests and scenarios are innovative and relevant from a research point of view. The members will be recruited from academic institutions and research organizations.
\par
Currently, the Research Advisory Board has no members yet.

\subsection{League Infrastructure}
In order to provide a forum for continuous discussions between teams and other stakeholders, the league builds and maintains an infrastructure consisting of web site, mailing lists, and repositories for documentation, software, and data. The infrastructure is complemented by a minimum infrastructure to be built and maintained by teams, i.e. teams should eventually create their own web page to which the RoboCup@Work League’s web pages can be linked.

\subsubsection{Infrastructure Maintained by the League}
The official website of RoboCup@Work is at

http://www.robocupatwork.org

This web site is the central place for information about the league. It contains general introductory information plus links to all other infrastructure components, such as a league wiki, the mailing lists, important documents such as this rule book, announcements of upcoming events as well as past events and participating teams.
\par
The league maintains several mailings lists:
\par
rc-work@lists.robocup.org
\par
This is the general RoboCup@Work mailing list. Anyone can subscribe, but a real name must be provided either as part of the email address or being specified on the mailing list subscription page. The list is moderated in order to avoid abuse by spammers. 
\par
rc-work-tc@lists.robocup.org
\par
This is the mailing list for the Technical Committee. Posts from non-members have to be approved by the list moderator. Approvals will be given only in well-justified cases.

\subsubsection{Infrastructure Maintained by Teams}
Each team is requested to build and maintain a minimum infrastructure for its team. This infrastructure consist of 

\begin{itemize}
	\item a team web site,
	\item a team contact, and
	\item a team mailing address.
\end{itemize}

The team web site should contain the following information:

\begin{itemize}
	\item name of the team, and team logo, if any
	\item affiliation of the team
	\item team leader including full contact information
	\item list of team members
	\item description of the team’s research interest and background
	\item description of specific approach pursued by the team
	\item description of the robot(s) used by the team
	\item list of relevant publications by team members

\end{itemize}

The team contact should be the official contact of the team. Usually, for university-based teams, this would be an academic person such as a professor or post-doc, who should, however, be responsive and be able to answer quickly when contacted by email.
\par
The team mailing address should be an email alias, which should be used to subscribe the team to the general RoboCup@Work mailing list. The email alias should at least include the team contact and the team leader.
